\chapter{Definici\'on de objetivos}\label{defobjetivos}

Para que podamos dar este proyecto por finalizado, debemos cumplir una serie de objetivos:

\begin{itemize}
	\item \textbf{Objetivos académicos:}
	\begin{itemize}
		\item Centralizar las distintas herramientas que ya existen en una sola aplicación web, de forma que el usuario disponga de toda la información en un mismo lugar.
		
		\item Crear como mínimo una herramienta nueva que aporte una funcionalidad útil de cara al videojuego.
		
		\item Permitir el registro de usuarios y el almacenaje de datos del mismo
		
		\item Evitar la reiterada introducción de datos de forma manual. Algunas herramientas necesitarán ciertos datos del jugador para realizar los cálculos, y dado que es una pesada tarea el introducirlos uno a uno varias veces, queremos intentar automatizarlo dentro de lo posible para que el usuario tenga que introducir los datos el mínimo numero de veces.
	\end{itemize}
	\item \textbf{Objetivos personales:}
	\begin{itemize}
		\item Hemos decidido realizar el apartado de front-end de este proyecto con la tecnología Angular, no solo debido a su utilidad para el desarrollo web de una sola página, sino también por su popularidad en el sector laboral, ya que pensamos que puede sernos útil de cara al futuro. Esta va a ser la primera vez que tratemos con Angular, por lo que pensamos que aprender a usar una tecnología desde cero y aplicarla de forma que se obtenga el mejor resultado posible es un gran reto que nos gustaría afrontar.
		\item Para el back-end hemos decidido usar Spring Boot, ya que como ibamos a empezar desde cero con Angular, pensamos que sería mejor usar algo que ya conocemos y que se nos ha estado enseñando durante gran parte de la carrera. Dicho esto, va a ser la primera vez que empecemos un proyecto desde cero sin contar con una guía, por lo que aunque conozcamos el lenguaje y la tecnología, es una buena oportunidad para poner en practica lo que hemos aprendido y demostrar que hemos asimilado los conocimientos.
	\end{itemize}
\end{itemize}






	
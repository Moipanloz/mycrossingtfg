\chapter{Definici\'on de objetivos}\label{defobjetivos}

Para que podamos dar este proyecto por finalizado, debemos cumplir una serie de objetivos:

\begin{itemize}
	\item \textbf{Objetivos académicos:}
	\begin{itemize}
		\item Centralizar varias de las distintas herramientas que ya existen en una sola aplicación web, de forma que el usuario disponga de toda la información en un mismo lugar y le resulte más cómodo, evitando el reiterado registro en distintos sitios web.
		
		\item Crear como mínimo una herramienta nueva que aporte una funcionalidad útil de cara al videojuego para poder así obtener al menos un factor diferenciante respecto a las demás aplicaciones.
		
		\item Facilitarle al jugador toda la información posible y relevante que pueda necesitar tanto para hacer uso de las herramientas como para su día a día dentro del juego, de forma que le sea más sencillo alcanzar sus objetivos y que pueda disponer de la información que necesite en un mismo sitio de manera bien organizada.
		
		\item Evitar que el usuario tenga que introducir datos de forma manual reiteradamente. Algunas herramientas necesitarán ciertos datos del jugador para realizar los cálculos, y dado que es una pesada tarea el introducirlos uno a uno varias veces, queremos intentar automatizarlo dentro de lo posible para que el usuario tenga que introducir los datos el mínimo numero de veces.
	\end{itemize}
	\item \textbf{Objetivos personales:}
	\begin{itemize}
		\item Realizar el apartado de front-end de este proyecto con la tecnología Angular, no solo debido a su utilidad para el desarrollo web de una sola página, sino también por su popularidad en el sector laboral, ya que pensamos que puede sernos útil de cara al futuro. Esta va a ser la primera vez que tratemos con Angular, por lo que pensamos que aprender a usar una tecnología desde cero y aplicarla de forma que se obtenga el mejor resultado posible es un gran reto que nos gustaría afrontar.
		
		\item Usar Spring Boot para el back-end, ya que como íbamos a empezar desde cero con Angular, pensamos que sería mejor usar algo que ya conocemos y que se nos ha estado enseñando durante gran parte de la carrera. Dicho esto, va a ser la primera vez que empecemos un proyecto desde cero sin contar con una guía, por lo que aunque conozcamos el lenguaje y la tecnología, es una buena oportunidad para poner en practica lo que hemos aprendido y demostrar que hemos asimilado los conocimientos.*\\
		
		\pagebreak
		
		*A medida que se ha seguido el desarrollo de la aplicación, dado que no conocíamos Angular ni el alcance que tiene, pensábamos que solo se encargaría de la parte visual, pero tras haber aprendido y desarrollado algunas funcionalidades, hemos visto que no solo se limita a lo visual, sino que también se encarga de gran parte de la lógica que hay por detrás.\\
		
		Debido a esto, llegamos a un punto en que lo único que necesitábamos era realizar una conexión a la base de datos desde Angular, y probando configuraciones hemos descubierto que usando XAMPP (MariaDB y Apache) se podía acceder a la base de datos mediante un archivo PHP que realizase la conexión y las peticiones oportunas, de forma que angular recogiera los datos mediante una petición HTTP (usando el back-end a modo de API).\\
		
		Es por esto que hemos optado por usar esta configuración en vez de usar Spring ya que, aunque es cierto que conocíamos el framework y el lenguaje, no nos resultó necesario usarlo ya que con XAMPP disponemos de todo lo que necesitábamos y conseguimos el objetivo principal que es la integración de los componentes de forma funcional para poder desarrollar sobre dicha configuración.\\
		
		Además, aunque hemos visto algo de PHP durante el curso, no ha sido mucho, por lo que se nos presenta otra oportunidad para ampliar nuestras capacidades y conocimientos investigando más sobre un lenguaje bastante usado actualmente, y aumentar nuestras habilidades como desarrolladores.\\
		
	\end{itemize}
\end{itemize}






	
\chapter{Introducción}\label{introduccion}

``Animal Crossing" {es} una saga de videojuegos publicado por Nintendo, cuyo objetivo es la simulación de vida. En todos los juegos, el usuario reside en un pueblo en el que además de él, habitan animales antropomorfos que hacen de vecinos, de forma que se pueda establecer una relación con ellos entablando conversaciones, realizando favores, celebrando eventos, etc.\\

Además, la mayor particularidad de este juego es que el tiempo y las estaciones transcurren al mismo tiempo que lo hacen en la vida real, de modo que si empezamos a jugar por la mañana, en el juego será de día, y si jugamos en invierno, el pueblo estará nevado. Esto da mucho juego ya que gran parte de las actividades que se pueden realizar, como la pesca o la caza de insectos, dependen mucho de la fecha y hora en la que se juegue. Pero no solo consiste en eso, sino que con cada entrega se han ido añadiendo funciones (muchas de las cuales se tratan en esta memoria) que hacen que el juego sea más y más completo cada vez.\\

Es por esto que en Internet podemos encontrar gran cantidad de herramientas para calcular diferentes aspectos y datos del último videojuego de la saga, \textit{``Animal Crossing: New Horizons”}. El problema recae en que muchas herramientas que podrían ser bastante útiles aún no han sido desarrolladas, así como la falta de un servicio web que recoja un compendio de las más importantes e interesantes. Esto hace perder bastante tiempo a todos aquellos jugadores que deseen consultar ciertos datos, ya que tendrían que navegar por distintos sitios, en vez de disponer de todo lo que necesitan en un mismo lugar.\\

Debido a esto, llegamos a la decisión de crear una aplicación web que contenga una colección de herramientas, así como de desarrollar algunas nuevas por nuestra cuenta, todo esto con una interfaz simple que permita acceder a cada una sin ninguna dificultad, y además intentando quedarnos dentro de la estética ya propuesta por el juego.\\

En nuestra aplicación, el usuario tiene a su alcance una serie de herramientas pensadas para los distintos ámbitos del juego: tareas personalizables diarias, un calendario para no perderse los posibles eventos del juego, una forma de predecir el mercado, un álbum de fotos personal, varios catálogos que cubren todo lo que el jugador puede encontrar en el juego, así como una forma para catalogar los ítems y disponer de una colección personal, etc.\\

Las herramientas y posibilidades son tan extensas que no podemos agruparlas todas, pero sí que podemos realizar un conjunto de las más importantes y de aquellas que le serán de más utilidad al usuario, y eso es lo que hemos hecho. Pensamos que es una aplicación muy útil, bastante ampliable de cara al futuro (ya que el juego sigue actualizándose con nuevo contenido), y que puede ser de gran ayuda para muchas personas que a día de hoy sigan disfrutando del juego.\\

Esta memoria consta de varios capítulos en los que tratamos diversos temas:

\begin{itemize}
	\item Capítulo 2: trata de los objetivos que nos hemos propuesto cumplir a la hora de realizar este proyecto.
	
	\item Capítulo 3: en este capítulo hablamos sobre algunas de las aplicaciones y herramientas similares ya disponibles, así como de las diferencias entre estas y nuestra aplicación.
	
	\item Capítulo 4: aquí tratamos la división del proyecto temporalmente, hablando sobre las distintas iteraciones así como de la previsión de horas y costes del proyecto frente a los datos reales.
	
	\item Capítulo 5: este capítulo es más técnico, ya que trata de los requisitos recogidos así como del diagrama de clases realizado y del primer boceto de lo que sería la aplicación.
	
	\item Capítulo 6: aquí hablamos sobre las tecnologías utilizadas, así como los principales problemas encontrados durante el desarrollo y las soluciones tomadas.
	
	\item Capítulo 7: otro capítulo algo más técnico donde recogemos el compendio de pruebas realizadas para garantizar que el proyecto funciona de manera adecuada.
	
	\item Capítulo 8: en este capítulo vemos la comparación de nuestra aplicación frente a otras ya disponibles, centrándonos en cada funcionalidad y comparándolas una a una.
	
	\item Capítulo 9: este capítulo es una guía para utilizar la aplicación en caso de que fuera necesario.
	
\end{itemize}
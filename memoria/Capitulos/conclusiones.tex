\chapter{Conclusiones y desarrollo futuro}\label{pruebas}

La realización del proyecto ha sido muy enriquecedora, ya que nos ha permitido comprender el esfuerzo que conlleva realizar un trabajo desde cero: pasar por cada una de sus fases, desde que solo es una idea, pasando por la planificación, el desarrollo, y hasta que finalmente es un producto completo y la idea ha pasado a ser realidad. Todo esto, sumado a la cantidad de preparación que requiere realizar un diseño y un análisis preciso, así como la dificultad de formar a un equipo en un lenguaje desconocido, conforma un reto bastante interesante del que nos sentimos bastante orgullosos de haber participado y llevado a cabo hasta el final.\\

Durante todo el trascurso del proyecto, como era de esperar, nos hemos visto en la necesidad de tomar varias decisiones y afrontar varios problemas (como se ha podido ver en el capítulo 6). Sin embargo, no podríamos estar más contentos con todos estos fallos y problemas que hemos tenido, ya que es así como de verdad podemos aprender para intentar evitar los mismos fallos en el futuro. Porque al final, este proyecto no se basaba exclusivamente en realizar una aplicación, sino que una de los principales objetivos era formarnos en el proceso y ampliar nuestro conocimientos, y es por eso por lo que decidimos usar un framework que no habíamos utilizado nunca, así como otros lenguajes en los que no teníamos tanta experiencia, todo para que pudiéramos aprovechar al máximo el tiempo empleado y salir del proyecto con nuevos conocimientos para abrirnos fronteras.\\

Y al final, eso es lo que hemos hecho, aprender. Aprender sobre el framework, sobre el lenguaje, sobre todo lo que necesitásemos para poder seguir adelante, y es algo que no hemos dejado de hacer durante toda la duración del proyecto. Sprint tras sprint descubríamos nuevas formas, mucho mas sencillas y eficientes, de abarcar los mismos problemas, y si a día de hoy (una vez finalizado el proyecto) echamos la vista hacia atrás, podemos ver todo lo que hemos aprendido desde que empezamos y el largo camino que hemos recorrido en algo menos de un año.\\

Cabe mencionar que consideramos que el proyecto no ha quedado tan completo como lo habríamos deseado, ni mucho menos, pero aspirar a más hubiera sido un grave error teniendo en cuenta el tiempo que teníamos asignado. Aún queda mucho margen de mejora, y a medida que pase el tiempo y vayan saliendo nuevas actualizaciones para el juego, más herramientas y más características se podrán añadir a nuestra aplicación. Sin embargo, seguimos pensando que el resultado obtenido ha sido bastante bueno (y bastante cercano en la gran mayoría de sus aspectos a lo que en un principio se ideó).\\

En resumen, no podríamos estar más satisfechos con el proyecto: tanto por el resultado que hemos acabado obteniendo, como por todo lo que hemos aprendido por el camino, sin duda ha sido una gran oportunidad para demostrar lo que hemos aprendido durante estos años de carrera, seguir formándonos y vernos cara a cara con un situación muy cercana a la realidad que sin duda nos será de gran ayuda para afrontar la siguiente etapa de nuestras vidas.\\

\textbf{\large Desarrollo futuro}\\

Aunque hemos realizado y centralizado muchas de las herramientas más importantes, un juego tan grande como es ``Animal Crossing" nos ofrece mucho margen de mejora para nuestra aplicación. Hay bastantes herramientas y funciones que nos hubiese gustado implementar, pero que por cuestión de tiempo hemos optado por no hacerlo para centrarnos en otras.\\

A continuación listamos algunas de las funcionalidades más relevantes que pensamos que podríamos implementar en un futuro como mejoras para la aplicación:

\begin{itemize}
	
	\item Modo anti-spoilers: una funcionalidad que consideramos bastante útil es un modo anti-spoilers para los listados de criaturas, ya que puede que haya ciertos jugadores que, a la llegada de un nuevo mes (y nuevas criaturas que capturar), quieran conocer pistas para saber sobre que momento del día y lugar pueden capturar aquellas criaturas que aún no tengan pero no deseen saber cuales son para llevarse la sorpresa, por lo que un modo que ocultase las imágenes y el nombre de las criaturas en el listado podría ser algo interesante para implementar.
	
	\item Al igual que los usuarios pueden subir su isla con su código de sueño, una funcionalidad muy interesante sería la de subir sus propios diseños. Dentro del juego el usuario puede crear diseños que puede usar tanto para vestimenta como para decoración, por lo que un catálogo de diseños donde los usuarios suban sus propias creaciones para compartirlas con el resto de jugadores, así como un sistema de favoritos y "Me gusta", es algo bastante útil para los jugadores.
	
	\item En el apartado de ``Colecciones especiales" se podría añadir un filtro de búsqueda por texto para facilitarle la búsqueda al usuario en aquellas colecciones especialmente extensas, así como añadir algunas otras posibles colecciones que puedan ser de interés para el usuario, ya que el juego sigue actualizándose y añadiendo nuevo contenido.
	
	\item Un apartado sobre el que se ha hecho poco hincapié ha sido el de la jardinería. El juego posee un sistema de cultivo de flores que, a partir de unos colores base se pueden llegar a generar otras flores de colores distintos y menos comunes. Es un sistema complejo ya que dependiendo del color y origen de la flor, ésta posee un gen u otro que puede servir para cultivar un color en particular. Pensamos que una herramienta bastante útil podría ser una especie de matriz sobre la que el usuario pueda colocar las flores que desee en la forma que así vea conveniente y que la aplicación le muestre qué colores pueden ser generados en ciertas posiciones de la matriz siguiendo la disposición introducida.
	
\end{itemize}



	


